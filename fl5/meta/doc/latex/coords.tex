\documentclass[a4paper, 11pt]{article}
\setlength{\parindent}{0em}
\setlength{\parskip}{1em}
\usepackage{amsmath}
\usepackage{mathtools}
\DeclarePairedDelimiter{\norm}{\lVert}{\rVert}
\usepackage{graphicx}
\renewcommand{\familydefault}{\sfdefault}
\newcommand*{\transp}[2][-3mu]{\ensuremath{\mskip1mu\prescript{\smash{\mathrm t\mkern#1}}{}{\mathstrut#2}}}%

\begin{document}

\title{Coordinate systems}
\author{techwinder}

\maketitle

\section{Coordinate systems}
\subsection{General}
A Boundary Element Method (BEM) typically requires the evaluation of weakly singular, singular, and hyper-singular integrals on panels enclosing the body. Such integrals are usually more conveniently analysed in a local frame of reference, such that the panel lies approximately in the $X-Y$ plane, or strictly in the case of flat panels as is considered in the present work. The origin $O$ and the $X$ and $Y$ axis can be chosen as is most convenient for the calculations.\\
The change of coordinate systems is therefore a basic operation of BEM, its principles are recalled hereafter.

\subsection{Cartesian coordinates}
\begin{figure}[h]
  \centering
  \includegraphics[scale=1.0]{figures/coords} 
  \caption{coordinate systems}
  \label{fig:triangle}
\end{figure} 
Let $(O,\vec{I}, \vec{J})$ be the absolute reference frame.  As is usual in BEM, this frame is chosen so that the panel lies in the (O,X,Y) plane. Let $O'(X_0, Y_0)$  denote the coordinates of $O'$ in the absolute reference frame.\\

A field point $P$ which has coordinates $(X,Y)$ in the global reference frame has coordinates $(x,y)$ in the local reference frame, such that:
\begin{equation}
\begin{bmatrix}  X  \\   Y    \\  \end{bmatrix} = \begin{bmatrix}  X_0  \\   Y_0    \\  \end{bmatrix} + 
\begin{bmatrix}
    \cos{\psi}    &-\sin{\psi} \\
   \sin{\psi}    & \cos{\psi} \\
\end{bmatrix}
\begin{bmatrix}  x  \\   y    \\  \end{bmatrix}
\end{equation}

The Jacobian of the transformation from global to local coordinates is:
\begin{equation}
\vert J \vert = \det\begin{bmatrix}
   \frac{\partial X}{\partial x} \  &\frac{\partial X}{\partial y} \\
   \frac{\partial Y}{\partial x} \  &\frac{\partial Y}{\partial y}
\end{bmatrix}
=1
\end{equation}

\subsection{2D Barycentric coordinates}
\begin{figure}[h]
  \centering
  \includegraphics[scale=1.0]{figures/barycentric} 
  \caption{Barycentric coordinates}
  \label{fig:barycentric} 
\end{figure}
Let a triangle be defined by three vertexes with coordinates $S_1(X_1, Y_1)$, $S_2(X_2, Y_2)$ and $S_3(X_3, Y_3)$, and let $S_1=O'$ as shown in Figure \ref{fig:barycentric}. \\ 
the triangle's area is $\vert e\vert = \frac{1}{2}\vert \det A\vert$, with
$$A = \begin{bmatrix}
1 &X_1 &Y_1\\
1 &X_2 &Y_2\\
1 &X_3 &Y_3\\
\end{bmatrix}$$ 

A point P(X,Y) inside the triangle defines three sub-triangles. \\Define the matrices:
$$A_1 = \begin{bmatrix}
1 &X &Y\\
1 &X_2 &Y_2\\
1 &X_3 &Y_3\\
\end{bmatrix} \qquad 
A_2 = \begin{bmatrix}
1 &X_1 &Y_1\\
1 &X &Y\\
1 &X_3 &Y_3\\
\end{bmatrix} \qquad
A_3 = \begin{bmatrix}
1 &X_1 &Y_1\\
1 &X_2 &Y_2\\
1 &X &Y\\
\end{bmatrix}
$$
The area of sub-triangle $T_i$ is $\vert e_i(P)\vert = \frac{1}{2}\vert\det{A_i}\vert$\par
Define the linear basis functions of the triangle represented in Fig. (\ref{fig:basis}) by:
\begin{equation}
b_i(P) = \frac{\vert e_i(P)\vert}{\vert e\vert} = \frac{\vert\det{A_i}\vert}{\vert\det{A}\vert}
\end{equation}
\begin{figure}[h]
  \centering 
  \includegraphics[scale=0.7]{figures/basis_functs}
  \caption{The linear basis functions of a triangle}
  \label{fig:basis}
\end{figure}
Since the sum of the areas of all three sub-triangles is the triangle's area, it follows that:
\begin{equation}
\forall P\in T \qquad b_1(P) + b_2(P) + b_3(P) = 1
\end{equation}

The functions $b_i$ are the linear basis functions of the triangle which are used in the Galerkin scheme. 


The Cartesian coordinates of a point $P(X,Y)$ inside the triangle can be converted to barycentric coordinates:
$$X_P = b_1(P)\; X_1 + b_2(P)\; X_2 + b_3(P)\; X_3$$
$$Y_P = b_1(P)\; Y_1 + b_2(P)\; Y_2 + b_3(P)\; Y_3$$
or in matrix form:
\begin{equation}
\begin{bmatrix} 1 \\ X_P \\ Y_P \end{bmatrix}
=\transp{A} 
\begin{bmatrix} b_1 \\ b_2 \\ b_3 \end{bmatrix}
\end{equation}

Any linear function of the panel's Cartesian coordinates $(X,Y)$ can also be written as a linear combination of the basis functions $(b_1, b_2, b_3)$:
\begin{equation} \label{eq:linearBasis}
\begin{split}
f(X,Y) &= \beta_0+\beta_X X+\beta_Y Y \\
       &= \beta_0(b_1+b_2+b_3) +\beta_X \big(b_1\; X_1 + b_2\; X_2 + b_3\; X_3\big) + \beta_Y \big(b_1\; Y_1 + b_2\; Y_2 + b_3\; Y_3 \big) \\
       &= (\beta_0 +\beta_X X_1+\beta_Y Y_1) b_1  +(\beta_0 +\beta_X X_2+\beta_Y Y_2) b_2  +(\beta_0 +\beta_X X_3+\beta_Y Y_3) b_3  \\
       &= \mu_1\,b_1 +\mu_2\,b_2 + \mu_3\,b_3
\end{split}
\end{equation}
The transformation from Cartesian to barycentric coefficients is therefore:
\begin{equation}\label{eq:barycentricConversion}
\begin{bmatrix}
\mu_1\\ \mu_2  \\ \mu_3
\end{bmatrix}
=
A
\begin{bmatrix}
\beta_0\\ \beta_X  \\ \beta_Y
\end{bmatrix}
\end{equation}


\section{Panel integrals}
\subsection{Source $-$ single layer integrals}
\underline{\textbf{Potential}}\\
The velocity potential induced at a field point $P$ by a source density $\sigma$ on a triangle $T$ is:
\begin{equation}
I_S=\iint_{T} \sigma \frac{-1}{r}\: dT 
\end{equation}
Since only flat panels and therefore constant source strengths are considered
\begin{equation}
I_S= -\sigma\iint_{T}  \frac{1}{r}\: dT
\end{equation}

\underline{\textbf{Velocity}}\\
The velocity vector at point P is the derivative of the potential at that point:\\
\begin{equation}
\vec{V}_S= -\sigma \:\nabla \big( \iint_{T} \frac{1}{r}\,dT  \big)= -\sigma \iint_{T} \nabla\frac{1}{r}\,dT = \sigma \iint_{T}  \frac{\vec{r}}{r^3}\,dT 
\end{equation}
which decomposes into 
\begin{equation}\label{eq:SourceVel}
\begin{split}
V_x = \sigma \iint_{T} \frac{(X_P-X)}{r^3}\: dX\,dY \\
V_y = \sigma \iint_{T} \frac{(Y_P-Y)}{r^3}\: dX\,dY\\ 
V_z = \sigma \iint_{T} \frac{Z_P}{r^3}\: dX\,dY\\
\end{split}
\end{equation}

\subsection{Doublet $-$ double layer integrals}
\underline{\textbf{Potential}}\\
The potential at point P induced by a doublet density $b(\vec{\rho})$ on the triangle $T$ is:
\begin{equation}
I_d = \iint_{T} b(\vec{\rho})\,\vec{n}\cdot\nabla{\frac{1}{\norm{\vec{R}-\vec{\rho}}}} \,dT 
\end{equation}
With
$$\vec{n}\cdot\nabla{\frac{1}{\norm{\vec{R}-\vec{\rho}}}} = \frac{\partial}{\partial Z}\frac{1}{r} = -\frac{Z_P}{r^3}$$
the potential becomes:
\begin{equation}\label{eq:potentialInt}
I_d = -\iint_{T} b(\vec{\rho})\, \frac{Z_P}{r^3} \,dT = -Z_P\iint_{T} b(\vec{\rho})\, \frac{1}{r^3} \,dT 
\end{equation} 
In the plane of the triangle therefore, the perturbation potential is zero everywhere except inside the triangle where it is undefined, but has two limits of opposite signs when approaching the panel by the upper or lower side.\par

In the present formulation, the doublet density is a linear function, which i.a.w. (\ref{eq:linearBasis}) can be expressed as a linear combination of the basis functions: 
\begin{equation}
\begin{split}
b(X,Y) &= \beta_1\; \mu_1(\vec{\rho}) + \beta_2\; \mu_2(\vec{\rho}) + \beta_3\; \mu_3(\vec{\rho}) \\
       &= \beta_0\, + \beta_x \; X + \beta_y \; Y
\end{split}
\end{equation}

Eq. \eqref{eq:potentialInt} reduces to a linear combination of the following elementary integrals
\begin{equation}\label{eq:integrals} 
\begin{split}
I_0^3 = \iint_{T} \frac{1}{r^3}\: dX\,dY\\
I_X^3 = \iint_{T} \frac{X}{r^3}\: dX\,dY\\
I_Y^3 = \iint_{T} \frac{Y}{r^3}\: dX\,dY\\ 
\end{split}
\end{equation}
\begin{equation}\label{eq:potentialIntShort} 
I_d = -Z_P\iint_{T} b(x,y)\,\ \frac{1}{r^3} \,dx\, dy = -Z_P\big(\beta_0\;I_0^3 + \beta_x\;I_X^3 + \beta_y\;I_Y^3\big)
\end{equation}


\begin{equation}
I_d =  -Z_P \big(A^{-1}
\begin{bmatrix}\mu_1\\ \mu_2  \\ \mu_3\end{bmatrix} \big)
\cdot
\begin{bmatrix} I_0^3\\ I_X^3  \\ I_Y^3\end{bmatrix}
\end{equation}

\underline{\textbf{Velocity}}\\
The velocity vector at point $P(\vec{R})$ is the derivative of the potential:\\
\begin{equation}
\vec{V}_d= \nabla{I_d}
\end{equation}
\begin{equation}\label{eq:Vxd}
\begin{split}
V_x =\: &Z_P\beta_0 \iint_{T} \frac{3(X_P-X)}{r^5}\: dX\,dY \\
     +&Z_P\beta_x \iint_{T} \frac{3X(X_P-X)}{r^5}\: dX\,dY  \\
     +&Z_P\beta_y \iint_{T} \frac{3Y(X_P-X)}{r^5}\: dX\,dY \\
\end{split}
\end{equation}
\begin{equation}\label{eq:Vyd} 
\begin{split}
V_y = &Z_P\beta_0 \iint_{T} \frac{3(Y_P-Y)}{r^5}\: dX\,dY \\
    +&Z_P\beta_x \iint_{T} \frac{3X(Y_P-Y)}{r^5}\: dX\,dY \\
    +&Z_P\beta_y \iint_{T} \frac{3Y(Y_P-Y)}{r^5}\: dX\,dY 
\end{split}
\end{equation}
\begin{equation}\label{eq:Vzd}
\begin{split}
V_z = &-\big(\beta_0\;I_0 + \beta_x\;I_x + \beta_y\;I_y\big) \\
&+3Z_P^2\bigg(\beta_0 \iint_{T} \frac{1}{r^5}\: dX\,dY 
+\beta_x \iint_{T} \frac{x}{r^5}\: dX\,dY 
+\beta_y \iint_{T} \frac{y}{r^5}\: dX\,dY \bigg)
\end{split}
\end{equation}

\subsection{Integral Evaluation}
\subsubsection{Integrals of interest}
The integrals which appear in the evaluation of the source and doublet potentials and their derivatives are of the form
\begin{equation}
I_{\alpha \beta}^n = \iint_{T} \frac{X^{\alpha}Y^{\beta}}{r^n} \,dX\, dY
\end{equation}
where $ i\in \left\lbrace  1,3,5 \right\rbrace$, and $\alpha, \beta \in \left\lbrace  1,2 \right\rbrace$.
\subsubsection{2D change of coordinate system}
Consider the sub-integral which appears in (\ref{eq:SourceVel})
\begin{equation}
I_X^3= \iint_{T} \frac{X}{r^3} \,dX\, dY
\end{equation}
where r is the distance from the field point $P$ to the surface element of the triangle.
\begin{equation}
r=\sqrt{(X_P-X)^2+(Y_P-X)^2+(Z_P-X)^2}
\end{equation}
Suppose that the triangle lies in the $(O,X,Y)$ plane. For the sake of example, define another frame $(O,x,y)$ in this plane, such that the new origin 0 has coordinates $(X_O, Y_O)$, and the x axis is rotated by $\psi$. \\
Given that $\vert J \vert=1$ and that distance $r$ is independent of the reference frame, $I_X^3$ and$I_Y^3$ can be written
\begin{equation}
I_X^3=\iint_{T}\frac{X_0+x\cos{\psi}-y\sin{\psi}} {r^3} \;dx\,dy
\end{equation}
\begin{equation}
I_Y^3=\iint_{T}\frac{Y_0+x\sin{\psi}+y\cos{\psi}} {r^3} \;dx\,dy
\end{equation}
This can be written in matrix form
\begin{equation}
\begin{bmatrix} I_X^3 \\ I_Y^3 \end{bmatrix} = \begin{bmatrix} X_0 \\ Y_0 \end{bmatrix} I_0^3
+\begin{bmatrix} &\cos{\psi}  &-\sin{\psi} \\ &\sin{\psi} &\cos{\psi} \end{bmatrix}\begin{bmatrix} I_x^3 \\ I_y^3 \end{bmatrix}
\end{equation}
where
$$I^3_0 = \iint_T \frac{1}{r^3}\; dx\, dy$$
$$I^3_x = \iint_T \frac{x}{r^3}\; dx\, dy$$
$$I^3_y = \iint_T \frac{y}{r^3}\; dx\, dy$$ 
The velocity vector induced by a unit uniform source density on the panel is
\begin{equation}
\begin{split}
V_x &= \iint_{T} \frac{(X_P-X)}{r^3}\: dX\,dY \\
    &= X_P\;I_0^3 - I_X  \\
    &= (X_P-X_0)\; I_0^3 - \cos{\psi}\; I_x^3 + \sin{\psi} \; I_y^3  \\
V_y &= (Y_P-Y_0)\; I_0^3 - \sin{\psi}\; I_x^3 - \cos{\psi} \; I_y^3  \\
V_z &= (Z_P-Z_0) \; I_0^3 = Z_P \; I_0^3;
\end{split}
\end{equation}

\subsubsection{3D}
Any change of reference frame in 3D can be written in the form
\begin{equation}
\begin{bmatrix} X \\ Y \\ Z \end{bmatrix} = \begin{bmatrix} X_0 \\ Y_0 \\ Z_0 \end{bmatrix}
+ \left[ A \right]  \begin{bmatrix} x \\ y \\ z \end{bmatrix}
\end{equation}
So that 
\begin{equation}
\begin{bmatrix} I_X^3 \\ I_Y^3 \\ I_Z^3 \end{bmatrix} = \begin{bmatrix} X_0 \\ Y_0 \\ Z_0 \end{bmatrix} I_0^3
+\left[ A \right] \begin{bmatrix} I_x^3 \\ I_y^3 \\ I_z^3 \end{bmatrix}
\end{equation}
The velocity potential is a scalar quantity which is independent of the reference frame.\\
The velocity vector induced by a unit uniform source density on the panel is
\begin{equation}
\begin{split}
V_x &= \iint_{T} \frac{(X_P-X)}{r^3}\: dX\,dY \\
    &= X_P\;I_0^3 - I_X^3  \\
    &= (X_P-X_0)\;I_0^3 - a_{00}I_x^3 - a_{01}I_y^3 - a_{02}I_z^3 \\
V_y &= (Y_P-Y_0)\;I_0^3 - a_{10}I_x^3 - a_{11}I_y^3 - a_{12}I_z^3 \\
V_z &= (Z_P-Z_0)\;I_0^3 - a_{20}I_x^3 - a_{21}I_y^3 - a_{22}I_z^3
\end{split}
\end{equation}





\subsection{Elementary integrals}
\begin{minipage}{\linewidth}
All integrals required in the 1st order BEM can be expressed as linear combinations of the following elementary integrals
$$I^1_0 = \iint_T \frac{1}{r}\; dx\, dy$$
$$I^1_x = \iint_T \frac{x}{r}\; dx\, dy$$
$$I^1_y = \iint_T \frac{y}{r}\; dx\, dy$$
$$I^3_0 = \iint_T \frac{1}{r^3}\; dx\, dy$$
$$I^3_x = \iint_T \frac{x}{r^3}\; dx\, dy$$
$$I^3_y = \iint_T \frac{y}{r^3}\; dx\, dy$$
$$I^5_0 = \iint_T \frac{1}{r^5}\; dx\, dy$$
$$I^5_x = \iint_T \frac{x}{r^5}\; dx\, dy$$
$$I^5_y = \iint_T \frac{y}{r^5}\; dx\, dy$$
$$I^5_{xx} = \iint_T \frac{x^2}{r^5}\; dx\, dy$$
$$I^5_{xy} = \iint_T \frac{xy}{r^5}\; dx\, dy$$
$$I^5_{yy} = \iint_T \frac{y^2}{r^5}\; dx\, dy$$
\end{minipage}
\end{document}






