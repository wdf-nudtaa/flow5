\documentclass[12pt]{article}
\usepackage[a4paper, portrait, margin=1in]{geometry}
\usepackage{mathtools}
\usepackage[export]{adjustbox}

\graphicspath{{./figures/}}
\renewcommand{\familydefault}{\sfdefault}


\begin{document}

\title{Numerical solution to the Orr-Sommerfeld equation using Keller's box method}
\author{techwinder}

\maketitle


\setlength{\parindent}{0pt}
\setlength{\parskip}{1em} 


\section{Orr-Sommerfeld equation} 
The Orr-Sommerfeld equation is expressed as (C\&C 5.3.16)
\begin{equation} \label{eq_OrrSommerfeld}
\phi^{iv} - \xi_1^2 \phi^{\prime\prime} - \xi_2^2(\phi^{\prime\prime}-\xi_1^2\phi) + \xi_3 \phi =0
\end{equation}
$\phi$ is the amplitude function of the vertical fluctuating velocity $v^\prime$

\section{Numerical solution}
The method proposed in C\&C section 5.6 is summarized as follows.

Equation (\ref{eq_OrrSommerfeld}) is expressed as a first order differential system by introducing new variables:
\begin{equation}
\begin{split}
f = \phi^\prime
f^\prime = s + \xi_1^2\phi
s^prime = g;
\end{split}
\end{equation}
so that eq. (\ref{eq_OrrSommerfeld}) is written as
\begin{equation}
g^\prime = \xi_2^2 s - \xi_3 \phi
\end{equation}
The boundary conditions become:
\begin{equation}
\begin{split}
y =0, \quad &\phi=0, \quad f=0 \\
y =\delta, \quad  &s+(\xi_1+\xi_2)f + \xi_1(\xi_1+\xi_2)\phi=0\\
 &g+\xi_2 s = 0 \\
\end{split}
\end{equation}

Consider a discretization of  the variable $y$ from in $J-1$ points $y_0$ to $y_{J-1}$

\end{document}













