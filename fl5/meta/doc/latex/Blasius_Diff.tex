\documentclass[12pt]{article}
\usepackage[a4paper, portrait, margin=1in]{geometry}
\usepackage{mathtools}
\usepackage[export]{adjustbox}
\usepackage{hyperref}

\graphicspath{{./figures/}}
\renewcommand{\familydefault}{\sfdefault}


\begin{document} 

\title{Blasius solution to the Falkner-Skan equation using 
Keller's box method}
\author{techwinder}

\maketitle 


\setlength{\parindent}{0pt}
\setlength{\parskip}{1em}



\section{Simplified Falkner-Skan equation}
Using the similarity variable 
$$\eta = y\sqrt{\frac{u_\infty}{\alpha\nu x}}$$
The simplified Falkner-Skan (FS) equation can be written:
\begin{equation} \label{eq:one}
f^{'''} + \frac{\alpha}{2} f(\eta)f^{''}(\eta) = 0
\end{equation}
where $\alpha$ is a parameter.
The boundary conditions are:
\begin{equation} \label{eq:oneBC}
f(0)=f^{'}(0)=0, \quad f^{'}(\infty) = 1
\end{equation}

\section{Keller's box}
The idea of Keller's box method is to
\begin{itemize}
  \item Reduce the differential equation to a set of first order equations by introducing new variables
  \item Discretize these equations using central differences
  \item Execute the following algorithm 
  \begin{itemize}
     \item Linearize the equations using first order approximation
     \item Initialize the unknowns with BC values where applicable and best guesses for the other components
     \item Rewrite the system of equations in matrix form, including the boundary conditions, where the unknown is the difference between values from iterations $n$ and $n+1$.
     \item Solve the system
     \item Update the set of values and check for convergence
  \end{itemize}
\end{itemize}

\section{Order reduction}
The equation is broken into three first order equations by introducing new variables $u$ and $v$ defined by:
\begin{equation}\label{eq:two}
\begin{split}
&\frac{df}{d\eta} = u \\
& \frac{du}{d\eta} = v
\end{split}
\end{equation}

FS equation becomes:
\begin{equation}\label{eq:three} 
\frac{dv}{d\eta}=-\frac{\alpha}{2}fv
\end{equation}

The boundary conditions given by equation \eqref{eq:oneBC} become:
\begin{equation}\label{eq:threeBC} 
f(0)=0, \qquad u(0)=0, \qquad u(\infty)=1
\end{equation}

\section{Discretization}
The equations $\eqref{eq:two}$ and $\eqref{eq:three}$  are discretized at $J$ points, using central differences in $\eta$. The spacing between points $j-1$ and $j$ is $h_j$.
\begin{equation}\label{eq:four} 
\begin{split}
\frac{f_j-f_{j-1}}{h_j}=u_{j-1/2}=\frac{1}{2}(u_j+u_{j-1}) \\
\frac{u_j-u_{j-1}}{h_j}=v_{j-1/2}=\frac{1}{2}(v_j+v_{j-1}) \\
\frac{v_j-v_{j-1}}{h_j}=-\frac{\alpha}{2}(fv)_{j-1/2}=-\frac{\alpha}{4} (f_jv_j+f_{j-1}v_{j-1})
\end{split}
\end{equation}
with $j\in[0,J-1]$.

\section{Newton linearization}
Equations $\eqref{eq:four}$ are linearized in the first order:
\begin{equation}\label{eq:five} 
\begin{split}
f_j^{n+1} = f_j^{n} + \delta f_j^{n} \\
u_j^{n+1} = u_j^{n} + \delta u_j^{n} \\
v_j^{n+1} = v_j^{n} + \delta v_j^{n} 
\end{split}
\end{equation}

where $n$ is the iteration number. The solution is deemed to have converged if all $\delta$ are less than a maximum given value.

Substituting the linearized values $\eqref{eq:five}$ in $\eqref{eq:four}$ gives:
\begin{equation}\label{eq:six} 
\begin{split}
f_j^n + \delta f_j^n -f_{j-1}^n - \delta f_{j-1}^n &= \frac{h_j}{2}\big(u_j^n+\delta u_j^n + u_{j-1}^n+\delta u_{j-1}^n \big)\\
u_j^n + \delta u_j^n -u_{j-1}^n - \delta u_{j-1}^n &= \frac{h_j}{2}\big(v_j^n+\delta v_j^n + v_{j-1}^n+\delta v_{j-1}^n \big)\\
v_j^n + \delta v_j^n -v_{j-1}^n - \delta v_{j-1}^n &= -\frac{\alpha h_j}{4}\big((f_j^n + \delta f_j^n)(v_j^n + \delta v_j^n) + (f_{j-1}^n + \delta f_{j-1}^n)(v_{j-1}^n + \delta v_{j-1}^n)\big)
\end{split}
\end{equation}
Neglecting second order terms, equations $\eqref{eq:six}$ can be rearranged as
\begin{equation}\label{eq:seven} 
\begin{split}
- \delta f_{j-1}^n + \delta f_j^n  - \frac{h_j}{2} \big(\delta u_{j-1}^n + \delta u_j^n \big) = r_j^n \\
- \delta u_{j-1}^n + \delta u_j^n  - \frac{h_j}{2} \big(\delta v_{j-1}^n + \delta v_j^n \big) = t_j^n \\
(-1+\frac{\alpha h_j}{4} f_{j-1}^n) \delta v_{j-1}^n + (1+\frac{\alpha h_j}{4}f_j^n) \delta v_j^n  + \frac{\alpha h_j}{4} \big(v_{j-1}^n \delta f_{j-1}^n +  v_j^n \delta f_j^n \big) = s_j^n
\end{split}
\end{equation}

with
\begin{equation}\label{eq:eigth} 
\begin{split}
r_j^n &= f_{j-1}^n - f_j^n + h_j u_{j-1/2}^n \\
t_j^n &= u_{j-1}^n - u_j^n + h_j v_{j-1/2}^n \\
s_j^n &= v_{j-1}^n - v_j^n - \frac{\alpha h_j}{2} (fv)_{j-1/2}^n 
\end{split}
\end{equation}

\section{Boundary conditions}
The BC given by equation \eqref{eq:oneBC} may be implemented in two ways:\par
\begin{enumerate}
  \item values at the boundary are enforced in the equations
  \item additional equations are introduced to account for the BC
\end{enumerate}
The second method is developed hereafter.\par


Note: A derivative boundary condition could be implemented in either of two ways:
\begin{enumerate}
  \item Backward difference at the boundary
$$\frac{\partial u}{\partial x}|_{i=N} = \frac{3u_N-4u_{N-1}+u_{N-2}}{2\Delta x}=0$$
i.e.
$$u_N = \frac{4u_{N-1}-u_{N-2}}{3}$$
  \item False boundary
$$\frac{\partial u}{\partial x}|_{i=N} = \frac{u_{N+1}-u_{N-1}}{\Delta x}=0$$
i.e.
$$u_{N+1} = u_{N-1}$$  
\end{enumerate}
In the present case the BC can be set directly on the variables and not on their derivatives.

\section{Matrix form}

The first two equations to consider are the translation of the boundary conditions $f(0)=0$ and $u(0)=0$.

If $f^0$ and $u^0$ are initialized with $f_0^0=0$ and $u_0^0=0$, then to enforce $f_0^1=...=f_0^n = 0$ and $u_0^1=...=u_0^n = 0$  the first two equations must be 
\begin{equation}
\begin{split}
\delta f_0^1 = ... = \delta f_0^n = 0\\
\delta u_0^1 = ... = \delta u_0^n = 0\\
\end{split}
\end{equation}
Similarly, the last equation to consider is the boundary condition $u(\infty)=1$, which, if $u^0$ is initialized with $u_{J-1}^0=1$, translates into 
\begin{equation}
\delta u_{J-1}^1 = ... = \delta u_{J-1}^n = 0\\
\end{equation}

The re-ordering of equations $\eqref{eq:seven}$ allows the problem to be written in matrix form:
\begin{equation}\label{eq:nine_a} 
A_j^n\delta_{j-1}^n + A_j^n\delta_{j}^n + C_j^n\delta_{j+1}^n = R_j^n
\end{equation}

Introducing the vector 
\begin{equation}
\Delta_j^n = \begin{bmatrix}
\delta f_j^n  \\
\delta u_j^n  \\
\delta v_j^n 
\end{bmatrix} 
\end{equation}

\begin{equation}\label{eq:nine} 
\begin{bmatrix}
B_0^n   &C_0^n  \\
A_1^n   &B_1^n   &C_1^n  \\
        &        &       &... \\
        &        &       &     &A_{J-2}^n  &B_{J-2}^n  &C_{J-2}^n   \\
        &        &       &     &           &A_{J-1}^n  &B_{J-1}^n   \\
\end{bmatrix} 
\begin{bmatrix}
\Delta_0^n \\
\Delta_1^n \\
... \\
\Delta_{J-2}^n \\
\Delta_{J-1}^n \\
\end{bmatrix}
=\begin{bmatrix}
R_0^n \\
R_1^n \\
... \\
R_{J-2}^n \\
R_{J-1}^n \\
\end{bmatrix}
\end{equation}

with
\begin{equation}
\begin{split}
A_j^n &= \begin{bmatrix}
-1                                &-\frac{h_j}{2}   &0 \\
\frac{\alpha h_j}{4} v_{j-1}^n    &0                &-1+\frac{\alpha h_j}{4} f_{j-1}^n \\
0                                 &0                &0
\end{bmatrix} \qquad 1 \leq j < J \\
B_j^n &= \begin{bmatrix}
1                                 &-\frac{h_j}{2}   &0 \\
\frac{\alpha h_j}{4} v_{j}^n      &0                &1+\frac{\alpha h_j}{4} f_{j}^n \\
0                                 &-1               &-\frac{h_{j+1}}{2}
\end{bmatrix} \qquad 1 \leq j < J-1 \\
C_j^n &= \begin{bmatrix}
0                                 &0                &0 \\
0                                 &0                &0 \\
0                                 &1                &-\frac{h_{j+1}}{2}
\end{bmatrix} \qquad 0 \leq j < J-1 \\
R_j^n &= \begin{bmatrix}
r_j^n  \\
s_j^n  \\
t_{j+1}^n 
\end{bmatrix} \qquad 1 \leq j < J-1 \\
\Delta_j^n &= \begin{bmatrix}
\delta_j^n  \\
\delta_j^n  \\
\delta_{j+1}^n 
\end{bmatrix} \qquad 0 \leq j < J \\
\end{split}
\end{equation}

The boundary conditions are implemented on the first and last rows:
\begin{equation}
\begin{matrix}
&B_0^n = \begin{bmatrix}
1                                 &0                &0 \\
0                                 &1                &0 \\
0                                 &-1               &-\frac{h_1}{2}
\end{bmatrix} 
&R_0^n = \begin{bmatrix}
0  \\
0  \\
t_{1}^n 
\end{bmatrix} \\
&B_{J-1}^n = \begin{bmatrix}
1                                   &-\frac{h}{2}   &0 \\
\frac{\alpha h_{J-1}}{4} v_{J-1}^n  &0                &1+\frac{\alpha h_{J-1}}{4} f_{J-1}^n \\
0                                   &1                &0
\end{bmatrix}
\qquad
&R_{J-1}^n = \begin{bmatrix}
r_{J-1}^n  \\
r_{J-1}^n  \\
0 
\end{bmatrix}
\end{matrix}
\end{equation}
Note that to prevent $B_0$ from being singular, the third row of the initial block matrix has been completed with the third equation of the "regular" block matrices. This implies that
\begin{itemize}
  \item In the "regular" matrices, the third row indices will be one more than those of the first two rows
  \item The last row of the last block matrices is available for the implementation of the BC $u(\infty)=0$
\end{itemize}
This \textit{ad hoc} ordering of the equations and of the BC allows to preserve the block tridiagonal structure with common definitions of the block matrices $A, B, C, R$.

\section{Algorithm}
The algorithm can be summarized as follows:
\begin{enumerate}
  \item Initialize $(f_j^0, u_j^0, v_j^0)$ with the BC and a best guess for the remaining values.
  \item Solve \eqref{eq:nine} using a matrix solver to get $(\delta f_j^n, \delta u_j^n, \delta v_j^n) $; the block-Thomas algorithm is recommended for tridiagonal matrices
  \item update $(f_j^n, u_j^n, v_j^n)  = (f_j^{n-1}, u_j^{n-1}, v_j^{n-1}) + (\delta f_j^n, \delta u_j^n, \delta v_j^n) $
  \item Iterate until all increment values $\delta f_j^n, \delta u_j^n, \delta v_j^n $ are less than a given value
\end{enumerate}

\newpage
\section{results}
The profiles for the functions $f(\eta)$, $u(\eta)$ and $v(\eta)$ are given in figures 1 to 3.
\begin{figure}
\centering
\includegraphics[width=12cm]{blasius-f}
\caption{$f(\eta)$}
\label{back_diff}
\end{figure}
\begin{figure}
\centering
\includegraphics[width=12cm]{blasius-u}
\caption{$u(\eta)$}
\label{back_diff}
\end{figure}
\begin{figure}
\centering
\includegraphics[width=12cm]{blasius-v}
\caption{$v(\eta)$}
\label{back_diff}
\end{figure}

\end{document}













