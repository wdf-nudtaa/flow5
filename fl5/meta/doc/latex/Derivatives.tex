\documentclass[a4paper, 11pt]{article}
\renewcommand{\familydefault}{\sfdefault}
\usepackage{mathtools}
\usepackage{color}
\usepackage{bm}
\usepackage{enumitem}
\usepackage{csquotes}

\usepackage{graphicx}
\graphicspath{{figures/}}
\renewcommand{\labelenumi}{\roman{enumi}}

\begin{document}

\title{Stability and control derivatives - implementation notes}
\author{techwinder}

\maketitle
   
\setlength{\parindent}{0pt}

\section{Introduction}
This document summarizes the method described in \cite{Drela} for the calculation of stability and control derivatives and its implementation in flow5.
\section{Motion}
The plane's velocity is denoted by $\vec{V}_\infty$ and its rotation rate by $\vec{\Omega}$. 
In body axes these vectors are expressed as
\begin{equation}
\vec{V}_\infty = Q_\infty \begin{Bmatrix}
           -\cos{\alpha}\cos{\beta} \\
            \sin{\beta} \\
           -\sin{\alpha}\cos{\beta}
           \end{Bmatrix} 
\end{equation}

\begin{equation}
\vec{\Omega} = \begin{Bmatrix}
           \Omega_x \\
           \Omega_y \\
           \Omega_z
           \end{Bmatrix}
\end{equation}

The angle of attack and the sideslip can be deduced from the freestream velocity (\cite{Drela} \textsection 9.5)
\begin{equation}
\begin{split}
\alpha &= \arctan{(V_z/V_x)} \\
\beta  &= \arctan(V_y/\sqrt{V_x^2+V_z^2})
\end{split}
\end{equation}

The velocity vector at a point $\vec{r}$ bound to the plane is
\begin{equation}
\vec{V}(\vec{r}) = \vec{V}_\infty + \vec{\Omega} \times \vec{r}
\end{equation}

The freestream potential associated to the translation motion is
\begin{equation}
\phi_\infty(\vec{r}) = \vec{V}_\infty \cdot \vec{r} = x\, V_x  + y \, V_y  + z\, V_z 
\end{equation}

Notet: the rotation motion of the fluid is not a potential flow.

\section{Axes}


\subsection{Stability axes}
Cf. \cite{Drela} \textsection 6.2.1.
\begin{figure}
\centering
\includegraphics[scale=0.35]{wind}
\caption{Wind direction}
\label{wind} 
\end{figure}

The rotation matrix to transform vectors from body axes to stability axes is

\begin{equation}
\bar{\bar{\mathbf{T}}}^s=
\begin{bmatrix}
\cos{\alpha}   &0    &\sin{\alpha}\\
0              &1    &0 \\
-\sin{\alpha}  &0    &\cos{\alpha}
\end{bmatrix} 
\end{equation}
Forces, moments and rotation rates are converted from body to stability axes using the following transformations:
\begin{equation}
\begin{Bmatrix} D   \\ Y   \\ L   \end{Bmatrix} 
= \bar{\bar{\mathbf{T}}}^s
\begin{Bmatrix} F_x   \\ F_y   \\ F_z   \end{Bmatrix} 
\end{equation}
\begin{equation}
\begin{Bmatrix} \mathcal{L}   \\ \mathcal{M}   \\ \mathcal{N}   \end{Bmatrix} 
= \bar{\bar{\mathbf{T}}}^s
\begin{Bmatrix} -M_x   \\ M_y   \\ -M_z   \end{Bmatrix} 
\end{equation}
\begin{equation}
\begin{Bmatrix} p   \\ q   \\ r   \end{Bmatrix} 
= \bar{\bar{\mathbf{T}}}^s
\begin{Bmatrix} -\Omega_x   \\ \Omega_y   \\ -\Omega_z   \end{Bmatrix} 
\end{equation}

\subsection{Wind axes}
The wind axes are deduced from stability axes by applying a rotation corresponding to the sideslip.

\begin{equation}
\bar{\bar{\mathbf{T}}}^sw=
\begin{bmatrix}
\cos{\alpha}\cos{\beta}   &-\sin{\beta}    &\sin{\alpha}\cos{\beta}\\
\sin{\alpha}\cos{\beta}   &\cos{\beta}     &\sin{\alpha}\sin{\beta}\\
-\sin{\alpha}              &0              &\cos{\alpha}
\end{bmatrix} 
\end{equation}


\section{Panel method}
\subsection{Indexes, subscripts and superscripts}
In this section, 
\begin{itemize}
		\item the subscript $_l$ indicates a control surface in the range $[0\cdot \cdot N_l-1]$
		\item the superscript $^k$ indicates the $k^{th}$ RHS; $k \in [u, v, w , p, q, r, 0\cdot \ldots N_l-1]$
		\item the subscripts $_i$ and $_j$ indicate panel indexes in the range $[0\ldots N-1]$
\end{itemize}
 
\subsection{Linearised solution}
\subsubsection{Boundary conditions}
The singularity strengths, i.e. vortex circulations or doublet densities, are determined so that the flow is tangent at each panel location. 
This requires to establish the expression of the velocity $\vec{V}_i$ and of the normal $\vec{n}_i$ at each panel's control point at the outset of the calculation.
\subsubsection{Velocity}
The velocity at any field point is the sum of the freestream and perturbation velocities. The perturbation is created by the bound and free vortices in the case of the VLM, and by the doublet and source singularities in the case of the panel methods.

Denote by $\mathbf{\Gamma}$  the unknown vector of singularity strengths on the panels, i.e. the vortex circulations in the case of the VLM and the doublet densities in the case of the panel method.

Denote by $\mathbf{\sigma}$ the source strengths on the panels in the case of the panel method. 

\begin{equation}
\label{velocity}
\vec{V}(\mathbf{r}) = -(\vec{V}_{\infty} + \vec{\Omega} \times \mathbf{r}) + \sum_j { \sigma_j  \vec{V}^\sigma_j(\mathbf{r})} + \sum_j { \Gamma_j  \vec{V}^\mu_j(\mathbf{r})}
\end{equation}
$\vec{V}^\sigma_j(\mathbf{r})$ is the velocity induced at point $\mathbf{r}$ by a unit source density located on panel $j$.

$\vec{V}^\mu_j(\mathbf{r})$ is the velocity induced by a unit doublet density or a vortex with unit circulation located on panel $j$.

\subsubsection{Panel normal}  
\begin{figure} 
\centering
\includegraphics[scale=0.3]{normal_derivative}
\caption{Normal derivative (copied from \cite{Drela} figure 6.8)}
\label{normal_derivative}
\end{figure}

Denote by $\delta_l$ the angular deflection of the control surface $l$.

The surface normal $\vec{n}_i$ at panel $i$ is calculated by linearising the small rotations of the control surfaces of which this panel is part.
\begin{equation}
\vec{n}_i = \vec{n}^0_i + \sum_l\frac{\partial\vec{n}_i}{\partial \delta_l} \delta_l= \vec{n}^0_i + \sum_l g_l \vec{h}_l \times \vec{n}^0_i
\end{equation}
where $g_l$ is the control's surface gain in radians.

\subsubsection{Source densities}
In accordance with the recommendation made in \cite{Maskew} and with the successful experience of xflr5 and flow5, the source strengths are defined to account for the freestream velocity jump across the surface. The uniform source strength on panel $i$ is therefore
\begin{equation}
\sigma_i = \frac{1}{4\pi} \big(\vec{V_\infty} + \vec{\Omega} \times \mathbf{r}\big)\cdot \vec{n}_i
\end{equation}
To optimize numerical robustness when solving the linear systems at the next stage, the source densities on the panels are expressed as a sum of terms corresponding to the components of the freestream velocity.
\begin{equation}
\sigma_i = V_x \, \sigma^u_i + V_y \, \sigma^v_i +  V_z \, \sigma^w_i  
         + \Omega_x \,  \sigma^p_i + \Omega_y \, \sigma^q_i + \Omega_z \, \sigma^r_i 
\end{equation}
\begin{equation}
\begin{split}
\sigma^u_i &= \frac{1}{4\pi} \vec{i} \cdot \vec{n}_i \\
\sigma^v_i &= \frac{1}{4\pi} \vec{j} \cdot \vec{n}_i \\
\sigma^w_i &= \frac{1}{4\pi} \vec{k} \cdot \vec{n}_i \\
\sigma^p_i &= \frac{1}{4\pi} (\vec{i} \times \mathbf{r}_i) \cdot \vec{n}_i \\
\sigma^q_i &= \frac{1}{4\pi} (\vec{j} \times \mathbf{r}_i) \cdot \vec{n}_i \\
\sigma^r_i &= \frac{1}{4\pi} (\vec{k} \times \mathbf{r}_i) \cdot \vec{n}_i 
\end{split}
\end{equation}

\subsubsection{Linear system}
At control point with index $i$ the flow tangency condition is
\begin{equation}
\begin{split}
\big[A_{ij}\big]\, \big[\Gamma_j\big] =
&V_x \big[ \vec{i} \cdot \vec{n}_i \big]+
 V_y \big[ \vec{j} \cdot \vec{n}_j \big]+
 V_z \big[ \vec{k} \cdot \vec{n}_k \big] \\
&+\Omega_x \big[ (\vec{i} \times \mathbf{r}_i) \cdot \vec{n}_i \big]+
  \Omega_y \big[ (\vec{j} \times \mathbf{r}_i) \cdot \vec{n}_i \big]+
  \Omega_z \big[ (\vec{k} \times \mathbf{r}_i) \cdot \vec{n}_i \big]\\
& - \sum_j { \sigma_j  \vec{V}^\sigma_j(\mathbf{r}_i)}\cdot \vec{n}_i \\
&\delta_1\big[-\vec{i} \cdot \vec{n}_{1_i}\big]+ \ldots + \delta_N\big[-\vec{i} \cdot \vec{n}_{N_i}\big]
\end{split}
\end{equation}


Given the linear structure of the equation, each RHS vector can be solved for independently:
\begin{equation}
\begin{split}
\mathbf{\Gamma^u} &= V_x \; \mathbf{A}^{-1} \bigg[\big(\vec{i} - \sum_j { \sigma^u_j  \vec{V}^\sigma_j(\mathbf{r}_i)}\big) \cdot \vec{n}_i\bigg] = V_x \mathbf{\Lambda^u}\\
\mathbf{\Gamma^v} &= V_y \; \mathbf{A}^{-1} \bigg[\big(\vec{j} - \sum_j { \sigma^v_j  \vec{V}^\sigma_j(\mathbf{r}_i)}\big) \cdot \vec{n}_i\bigg] = V_x \mathbf{\Lambda^v}\\
\mathbf{\Gamma^w} &= V_z \; \mathbf{A}^{-1} \bigg[\big(\vec{k} - \sum_j { \sigma^w_j  \vec{V}^\sigma_j(\mathbf{r}_i)}\big) \cdot \vec{n}_i\bigg] = V_x \mathbf{\Lambda^w}\\
\mathbf{\Gamma^p} &= \Omega_x \mathbf{A}^{-1} \bigg[\big(\vec{i} \times \mathbf{r}_i - \sum_j { \sigma^p_j  \vec{V}^\sigma_j(\mathbf{r}_i)}\big)\cdot \vec{n}_i\bigg] 
                                      = \Omega_x \mathbf{\Lambda^p}\\
\mathbf{\Gamma^q} &= \Omega_y \mathbf{A}^{-1} \bigg[\big(\vec{i} \times \mathbf{r}_j - \sum_j { \sigma^q_j  \vec{V}^\sigma_j(\mathbf{r}_i)}\big)\cdot \vec{n}_i\bigg] 
                                      = \Omega_y \mathbf{\Lambda^q}\\
\mathbf{\Gamma^r} &= \Omega_z \mathbf{A}^{-1} \bigg[\big(\vec{j} \times \mathbf{r}_k - \sum_j { \sigma^r_j  \vec{V}^\sigma_j(\mathbf{r}_i)}\big)\cdot \vec{n}_i\bigg] 
                                      = \Omega_z \mathbf{\Lambda^r}\\
\mathbf{\Gamma^k} &= \delta_k \mathbf{A}^{-1} \bigg[-\vec{k} \cdot \vec{n}_{k_i}\bigg]
                                      = \delta_k \mathbf{\Lambda^k}\\
 &k\in[0\ldots N_l-1]
\end{split}
\end{equation}
The general solution is a linear combination of the elementary solutions.
\begin{equation}
\mathbf{\Gamma} = V_x \mathbf{\Lambda^u} + V_y \mathbf{\Lambda^v} + V_z \mathbf{\Lambda^w}  \Omega_x \mathbf{\Lambda^p} + \Omega_y \mathbf{\Lambda^q} + \Omega_z \mathbf{\Lambda^r} + \sum_k \delta_k \mathbf{\Lambda^k}
\end{equation}

\subsubsection{Induced velocity, force and moment}
Let $\Gamma^k_j$ be the bound vorticity on panel $_j$ associated to solution $^k$, with $k\in \lbrace u, v, w, p, q, r \rbrace$

In the case of the VLM, the force acting on panel $_i$ induced by the set of panel vorticities $\Gamma^k_j$ associated to the $k^{th}$ solution is deduced from the Kutta-Joukowski theorem.
\begin{equation}
\vec{F}_i^k(\mathbf{r}) = \rho \, \Gamma_i^k \, \vec{V}(\mathbf{r}) \times  \vec{dl}_i
\end{equation}
where the velocity vector $\vec{V}(\mathbf{r})$ is given by \eqref{velocity}.

In the case of the panel methods, the panel forces are deduced from the local pressure coefficient. See \cite{Maskew} for more detailed explanations.

The moment of this force acting at a reference point $\mathbf{G}$ is
\begin{equation}
\vec{M}_i^k(\mathbf{r}) = (\mathbf{r}-\mathbf{G}) \times \vec{F}_i^k(\mathbf{r})  
\end{equation}
The total force and moment associated to solution $k$ are recovered by summation over the panels
\begin{equation}
\begin{split}
\vec{F}^k(\mathbf{r}) =  \sum_i \vec{F}_i^k(\mathbf{r}) \\
\vec{M}^k(\mathbf{r}) =  \sum_i \vec{M}_i^k(\mathbf{r})
\end{split}
\end{equation}

\subsection{Force derivatives}

\subsubsection{$\alpha$ derivative}
\begin{equation}
\frac{\vec{F}^k(\mathbf{r})}{\partial \alpha} =  \sum_i \frac{\vec{F}_i^k(\mathbf{r}) }{\partial \alpha}
\end{equation}
The derivative of the force acting on panel $i$ is
\begin{equation}
\label{F_ki}
\frac{\partial \vec{F}^k_i}{\partial \alpha} = \rho 
(\Gamma^k_i\frac{\partial \vec{V}}{\partial \alpha} 
  +  \frac{\partial \Gamma^k_i}{\partial \alpha} \vec{V}) \times \vec{dl}_i
\end{equation}
\begin{equation}
\begin{split}
\frac{\partial \vec{V}}{\partial \alpha} &= \frac{\partial}{\partial \alpha}\big(-(\vec{V}_{\infty} + \vec{\Omega} \times \mathbf{r}) + \sum_j { \Gamma^k_j  \vec{V}_j}\big) \\
     &= -\frac{\partial \vec{V_{\infty}}}{\partial \alpha} + \sum_j { \frac{\Gamma^k_j}{\partial \alpha}  \vec{V}_j} 
\end{split}
\end{equation}


\begin{equation}
\frac{\partial \Gamma^k_i}{\partial \alpha} =  \begin{cases}  
  \Lambda^u_i \frac{\partial V_x}{\partial \alpha}  & \quad \text{if } k=u \\
  \Lambda^w_i \frac{\partial V_z}{\partial \alpha}  & \quad \text{if } k=w \\
  0             & \quad \text{otherwise } 
  \end{cases}  
\end{equation}

and finally
\begin{equation}
\frac{\partial \vec{V_{\infty}}}{\partial \alpha} =  \begin{Bmatrix}
\sin{\alpha} \cos{\beta} \\
0    \\
-\cos{\alpha}  \cos{\beta}
\end{Bmatrix} 
\end{equation}


\subsubsection{$\beta$ derivative}

\begin{equation}
\frac{\vec{F}^k(\mathbf{r})}{\partial \beta} =  \sum_i \frac{\vec{F}_i^k(\mathbf{r}) }{\partial \beta}
\end{equation}
\begin{equation}
\frac{\partial \vec{F}^k_i}{\partial \beta} = \rho 
(\Gamma^k_i\frac{\partial \vec{V}}{\partial \beta} 
  +  \frac{\partial \Gamma^k_i}{\partial \beta} \vec{V}) \times \vec{dl}_i
\end{equation}
\begin{equation}
\begin{split}
\frac{\partial \vec{V}}{\partial \beta} &= \frac{\partial}{\partial \beta}\big(-(\vec{V}_{\infty} + \vec{\Omega} \times \mathbf{r}) + \sum_j { \Gamma^k_j  \vec{V}_j}\big) \\
     &= -\frac{\partial \vec{V_{\infty}}}{\partial \beta} + \sum_j { \frac{\Gamma^k_j}{\partial \beta}  \vec{V}_j} 
\end{split}
\end{equation}
\begin{equation}
\frac{\partial \Gamma^k_i}{\partial \beta} =  \begin{cases}  
  \Lambda^v_i \frac{\partial V_y}{\partial \beta}  & \quad \text{if } k=v \\
  0             & \quad \text{otherwise } 
  \end{cases}  
\end{equation}
\begin{equation}
\frac{\partial \vec{V_{\infty}}}{\partial \beta} =
  \begin{Bmatrix}   \cos{\alpha} \sin{\beta} \\   \cos{\beta}    \\   \sin{\alpha}  \sin{\beta}   \end{Bmatrix} 
\end{equation}

\subsubsection{u derivative}

\begin{equation}
\frac{\vec{F}^k(\mathbf{r})}{\partial u} =  \sum_i \frac{\vec{F}_i^k(\mathbf{r}) }{\partial u}
\end{equation}
\begin{equation}
\frac{\partial \vec{F}^k_i}{\partial u} = \rho 
(\Gamma^k_i\frac{\partial \vec{V}}{\partial u} 
  +  \frac{\partial \Gamma^k_i}{\partial u} \vec{V}) \times \vec{dl}_i
\end{equation}
\begin{equation}
\begin{split}
\frac{\partial \vec{V}}{\partial u} &= \frac{\partial}{\partial u}\big(-(\vec{V}_{\infty} + \vec{\Omega} \times \mathbf{r}) + \sum_j { \Gamma^k_j  \vec{V}_j}\big) \\
     &= -\frac{\partial \vec{V_{\infty}}}{\partial u} + \sum_j { \frac{\Gamma^k_j}{\partial u}  \vec{V}_j} 
\end{split}
\end{equation}
\begin{equation}
\frac{\partial \Gamma^k_i}{\partial u} =  \begin{cases}  
  \Lambda^u_i  & \quad \text{if } k=u \\
  0            & \quad \text{otherwise } 
  \end{cases}  
\end{equation}
\begin{equation}
\frac{\partial \vec{V_{\infty}}}{\partial u} =
  \begin{Bmatrix}  1 \\   0   \\   0   \end{Bmatrix} 
\end{equation}


\subsection{Moment derivatives}
\subsubsection{$\alpha$ derivative}

\begin{equation}
\frac{\vec{M}^k(\mathbf{r})}{\partial \alpha} =  \sum_i \frac{\vec{M}_i^k(\mathbf{r}) }{\partial \alpha}
\end{equation}
\begin{equation}
\begin{split}
\frac{\partial \vec{M}^k_i}{\partial \alpha} &= \rho \frac{\partial}{\partial \alpha} \bigg( 
  (\mathbf{r}-\mathbf{G}) \times \vec{F}^k_i(\mathbf{r}) \bigg) \\
  &= \rho  (\mathbf{r}-\mathbf{G}) \times \frac{\partial \vec{F}^k_i(\mathbf{r})}{\partial \alpha}
\end{split}
\end{equation}
where the derivative of the panel's force is given by equation \eqref{F_ki}.

\subsubsection{u derivative}
\begin{equation}
\frac{\vec{M}^k(\mathbf{r})}{\partial u} =  \sum_i \frac{\vec{M}_i^k(\mathbf{r}) }{\partial u}
\end{equation}
\begin{equation}
\begin{split}
\frac{\partial \vec{M}^k_i}{\partial u} &= \rho \frac{\partial}{\partial u} \bigg( 
  (\mathbf{r}-\mathbf{G}) \times \vec{F}^k_i(\mathbf{r}) \bigg) \\
  &= \rho  (\mathbf{r}-\mathbf{G}) \times \frac{\partial \vec{F}^k_i(\mathbf{r})}{\partial u}
\end{split}
\end{equation}


\subsection{Control derivatives}
\subsubsection{Translation derivative}
\begin{equation}
\frac{\vec{F}^k(\mathbf{r})}{\partial g_l} =  \sum_i \frac{\vec{F}_i^k(\mathbf{r}) }{\partial g_l}
\end{equation}
\begin{equation}
\frac{\partial \vec{F}^k_i}{\partial g_l} = \rho 
(\Gamma^k_i\frac{\partial \vec{V}}{\partial g_l} 
  +  \frac{\partial \Gamma^k_i}{\partial g_l} \vec{V}) \times \vec{dl}_i
\end{equation}
\begin{equation}
\begin{split}
\frac{\partial \vec{V}}{\partial g_l} &= \frac{\partial}{\partial g_l}\big(-(\vec{V}_{\infty} + \vec{\Omega} \times \mathbf{r}) + \sum_j { \Gamma^k_j  \vec{V}_j}\big) \\
     &= \sum_j { \frac{\Gamma^k_j}{\partial g_l}  \vec{V}_j} 
\end{split}
\end{equation}

\begin{equation}
\frac{\partial \Gamma^k_i}{\partial g_l} =  \begin{cases}  
  \Lambda^l_i   & \quad \text{if } k=l \\
  0             & \quad \text{otherwise } 
  \end{cases}  
\end{equation}


\subsubsection{Angular derivative}
\begin{equation}
\frac{\vec{M}^k(\mathbf{r})}{\partial g_l} =  \sum_i \frac{\vec{M}_i^k(\mathbf{r}) }{\partial g_l}
\end{equation}
\begin{equation}
\begin{split}
\frac{\partial \vec{M}^k_i}{\partial g_l} &= \rho \frac{\partial}{\partial g_l} \bigg( 
  (\mathbf{r}-\mathbf{G}) \times \vec{F}^k_i(\mathbf{r}) \bigg) \\
  &= \rho  (\mathbf{r}-\mathbf{G}) \times \frac{\partial \vec{F}^k_i(\mathbf{r})}{\partial g_l}
\end{split}
\end{equation}

\newpage
\begin{thebibliography}{9}

\bibitem{Drela}
Mark Drela, \textit{Flight Vehicle Aerodynamics}, Massachusetts Institute of Technology, 2014.
\bibitem{Etkin
} 
B. Etkin and L. Reid, \textit{Dynamics of Flight}, 3rd Edition, 1996.

\bibitem{Maskew} 
Brian Maskew, \textit{VSAERO Theory Document. A Computer Program for Calculating Nonlinear Aerodynamic Characteristics of Arbitrary Configurations}, NASA Contractor report 4023, September 1987.


\end{thebibliography}

\end{document}
 












