\documentclass[12pt]{article}
\usepackage[a4paper, portrait, margin=1in]{geometry}
\usepackage{mathtools}
\usepackage[export]{adjustbox}
\usepackage{color}
\usepackage{amsmath}
\usepackage{hyperref}
\usepackage{xfrac}

\graphicspath{{./figures/}}
\renewcommand{\familydefault}{\sfdefault}

 
\begin{document}

\title{Differential solution method for the Inverse Boundary Layer problem with the CS turbulent flow model - Implementation notes}
\author{techwinder}

\maketitle


\setlength{\parindent}{0pt}
\setlength{\parskip}{1em}

\section{Introduction}
This document is a summary of the notes taken during the implementation in flow5 of a flow solver for boundary layers around airfoils. 

flow5 implements the boundary layer models developed by T. Cebeci in references \cite{CC} to \cite{TC2}.


\section{BL Differential equations} 
\subsection{Equation C\&C 4.5.1}
\numberwithin{equation}{section}

The momentum and continuity equations for flows with a pressure gradient are expressed as (C\&C 4.5.8c) :
\begin{equation}
f^{\prime\prime\prime} + \frac{m+1}{2} ff^{\prime\prime} + m \big(1-f^{\prime 2}\big) = x \big( u \frac{\partial u}{\partial x}  - v \frac{\partial f}{\partial x}  \big)
\end{equation}
This equation with its boundary conditions is solved in differential form using Keller's box method as described in C\&C paragraph 4.5.

$m$ is a dimensionless pressure gradient parameter:
$$m = \frac{x}{u_e} \frac{du_e}{dx}$$

\newpage

\subsection{Cebeci-Smith turbulence model}
The Cebeci-Smith turbulence model is a formulation widely used for wall boundary-layer flows. It models the Reynolds shear stress in the momentum equations using algebraic eddy viscosity (C\&C paragraph 6.3.1). This formulation is a two layer approach, which uses eddy viscosity for both the inner and outer layers.

\underline{In the inner region:}
\begin{equation}
(\nu_t)_i = l^2  \mathopen| \frac{\partial u}{\partial y} \mathclose| \gamma_{tr} \qquad 0\leq y \leq y_c
\end{equation}


\underline{In the outer region}
\begin{equation}
(\nu_t)_o = \alpha \int_0^\delta (u_e-u) dy \, \gamma_{tr} \gamma = \alpha u_e \delta^*  \, \gamma_{tr} \gamma  \qquad  y_c\leq y \leq \delta
\end{equation}

The eddy viscosity is included in the term $b$ (TC 8.2.8):
\begin{equation}
b= 1+\frac{\varepsilon_m}{\nu}
\end{equation}

For a turbulent flow over a flat plate with no pressure gradients, the Cebeci-Smith eddy viscosities are illustrated in figure \ref{CC_fig611b}. 

\begin{figure}
\centering
\includegraphics[width=15cm]{CS_turb}
\caption{Eddy viscosity over a flat plate with no pressure gradient}
\label{CC_fig611b}
\end{figure}

\newpage
\begin{thebibliography}{9}
\bibitem{CC} 
Tuncer Cebeci and Jean Cousteix. 
\textit{Modeling and computation of boundary layer flows}, Springer, 2005.
 
\bibitem{TC} 
Tuncer Cebeci. 
\textit{Analysis of turbulent flows with computer programs}, Elsevier, 2013.

\bibitem{TC2} 
Tuncer Cebeci. 
\textit{An engineering approach to the calculation of aerodynamic flows}, Springer, 1999.

\bibitem{MD1} 
Mark Drela. 
\textit{XFoil: An analysis and design system for low Reynolds number airfoils}. 
Low Reynolds number aerodynamics, Springer-Verlag, Lec. notes in Eng. 54. 1989.

\bibitem{MD2} 
Mark Drela and Michael Giles. 
\textit{Viscous-Inviscid analysis of transonic and low Reynolds number airfoils}. 
AIAA Journal, Vol. 25, No. 10 (1987), pp. 1347-1355.


 \end{thebibliography}

\end{document}





