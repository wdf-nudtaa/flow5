\documentclass[12pt]{article}
\usepackage[a4paper, portrait, margin=1in]{geometry}
\usepackage{mathtools}
\usepackage[export]{adjustbox}
\usepackage{color}
\usepackage{amsmath}
\usepackage{hyperref}
\usepackage{xfrac}

\graphicspath{{./figures/}}
\renewcommand{\familydefault}{\sfdefault}

 
\begin{document}

\title{Implementation of Eppler's BL method}
\author{techwinder}

\maketitle


\setlength{\parindent}{0pt}
\setlength{\parskip}{1em}
\section{Solution method}
The method described in ref. \cite{Eppler} is a 2-equation integral method, using the integral momentum  equation
\begin{equation} \label{eq_momentum}
\frac{d\theta}{dx} + (2+H) \, \frac{dU_e}{dx} \frac{\theta}{U_e} = C_f
\end{equation}
and the kinetic energy equation 
\begin{equation}\label{eq_energy}
\frac{d\lambda}{dx} + 3 \, \frac{dU_e}{dx} \frac{\lambda}{U_e} = C_D
\end{equation}

where
  \begin{itemize}
     \item $H$ is the kinematic shape parameter 
     \item $\theta$ is the momentum thickness
     \item $\lambda$ is the kinetic thickness (usually noted as $\delta_3$; the variable $\lambda$ is chosen to alleviate subscripts)
     \item $U_e$ is the edge velocity.
     \item $C_f$ is the friction coefficient
     \item $C_D$ is the dissipation coefficient
  \end{itemize}

Note that in \cite{Eppler} and in the equation above, the friction coefficient $C_f$ is
\begin{equation}
C_f = \frac{\tau_0}{\rho U_e^2}
\end{equation}
which is twice the usual definition.

The primary unknowns chosen in ref. \cite{Eppler} are $\theta$ and the kinetic energy shape parameter $H^*$ defined by 
$$H^* = \frac{\lambda}{\delta^*}$$
However, since eq. (\ref{eq_energy}) is expressed in terms of $\lambda$, this variable is selected with $\theta$ to solve the non linear system made of equations (\ref{eq_momentum}) and (\ref{eq_energy}).

The quantities $H$, $H^*$, $C_f$ and $C_D$ can be expressed in terms of $\theta$ and $\lambda$ using equations (46) through (50) of reference \cite{Eppler}.

If the solution at a given streamwise station with index $0$ is known, the solution at the next station is obtained using a central difference (Crank-Nicolson) scheme. Let
\begin{equation}
\begin{split}
f(\theta, \lambda) = \frac{\theta-\theta_0}{dx} + (2+H_{1/2}) \frac{u-u_0}{dx} \frac{\theta_{1/2}}{u_{1/2}} - C_{f\,1/2} \\
g(\theta, \lambda) = \frac{\lambda-\lambda_0}{dx} + 3 \frac{u-u_0}{dx} \frac{\lambda_{1/2}}{u_{1/2}} - C_{D\,1/2} 
\end{split}
\end{equation}
The subscript $_{1/2}$ refers to midpoint quantities, e.g.
\begin{equation}
H_{1/2} = \frac{H_0+H}{2}
\end{equation}

The discretized system 
\begin{equation}\label{eq_discretized}
\begin{split}
f(\theta, \lambda) = 0\\
g(\theta, \lambda) = 0
\end{split}
\end{equation}
 is non linear and is solved using  a Newton-Raphson scheme in the form
\begin{equation}
\begin{bmatrix}
\frac{\partial f}{\partial \theta}   &\frac{\partial f}{\partial \lambda}  \\
\frac{\partial g}{\partial \theta}   &\frac{\partial g}{\partial \lambda}  
\end{bmatrix} 
\begin{bmatrix}
\Delta \theta \\
\Delta \lambda
\end{bmatrix} 
 = 
 \begin{bmatrix}
-f\\
-g
\end{bmatrix} 
\end{equation}
\begin{equation}
\begin{split}
\theta^{n} = \theta^{n-1}  +\Delta \theta \\
\lambda^{n} = \lambda^{n-1}  +\Delta \lambda \\
\end{split}
\end{equation}
The subscript $^n$ refers to the iteration number.
The variables can be initialized with Blasius's solution, or with the quantities at station $0$.
The convergence criterion is defined as a maximum value of  $\vert f \vert$ and $\vert g \vert$.


\newpage
\begin{thebibliography}{9}
\bibitem{Eppler} 
Richard Eppler, et al.
\textit{A Computer Program for the Design and Analysis of Low-Speed Airfoils}. 
NASA TM 80210, August 1980.


 \end{thebibliography}

\end{document}





